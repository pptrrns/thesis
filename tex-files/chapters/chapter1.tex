\begin{savequote}[85mm]
	\small{Nulla facilisi. In vel sem. Morbi id urna in diam dignissim feugiat. Proin molestie tortor eu velit. Aliquam erat volutpat. Nullam ultrices, diam tempus vulputate egestas, eros pede varius leo.
		\qauthor{Quoteauthor Lastname}}
\end{savequote}

\chapter{Lorem Ipsum}
\label{chap:introduction}

\newthought{A beautiful} \LaTeX\ templatefor a thesis/dissertation at the \emph{Instituto Tecnológico Autónomo de México} (ITAM). The template is based on the \href{https://github.com/suchow/Dissertate}{Dissertate} template and includes everything needed to support the production and typesetting of a B.A./B.S. dissertation. To get started, follow these steps: (\emph{i}) install \LaTeX\; (\emph{ii}) install the default fonts: EB Garamond, Lato, and Source Code Pro. The font files are provided in the folder \texttt{fonts}; (\emph{iii}) personalize the document by filling out your information in \texttt{frontmatter/personalize.tex} and in \texttt{endmatter/personalize.tex}; (\emph{iv}) customize the abstract, preface, and acknowledgments in the \texttt{frontmatter} folder; (\emph{v}) edit the files located in \texttt{chapters} to add the content of your thesis (chapters and appendices); (\emph{vi}) build your thesis with \texttt{XeLaTeX}.

The \texttt{references.bib} file contains the references. By default, the references are cited in the \href{https://connect.apsanet.org/stylemanual}{APSA} style. Use the command \texttt{\string\citep\{\}} to manage the references.

\texttt{\string\citep\{sen1995, angrist2015\}} $\longrightarrow$ \citep{sen1995, angrist2015}

\texttt{\string\citep[pg.\~32]\{sen1977\}} $\longrightarrow$ \citep[pg.~32]{sen1977}

\texttt{\string\citep\{sen1995\}} $\longrightarrow$ \citep{sen1995}

\texttt{\string\citep[véase][pp.\~5]\{sen1995\}} $\longrightarrow$ \citep[\emph{véase}][pg.~5]{sen1995}

\texttt{\string\citep[Chap.\~5]\{angrist2015\}} $\longrightarrow$ \citep[Chap.~5]{angrist2015}
